\documentclass[10pt,-letter paper]{article}
\usepackage[left=1in, right=0.75in, top=1in, bottom=0.75in]{geometry}
\usepackage{graphicx} % Required for inserting images
\usepackage{siunitx}
\usepackage{setspace}
\usepackage{gensymb}
\usepackage{xcolor}
\usepackage{caption}
%\usepackage{subcaption}
\doublespacing
\singlespacing
\usepackage[none]{hyphenat}
\usepackage{amssymb}
\usepackage{relsize}
\usepackage[cmex10]{amsmath}
\usepackage{mathtools}
\usepackage{amsmath}
\usepackage{commath}
\usepackage{amsthm}
\interdisplaylinepenalty=2500
%\savesymbol{iint}
\usepackage{txfonts}
%\restoresymbol{TXF}{iint}
\usepackage{wasysym}
\usepackage{amsthm}
\usepackage{mathrsfs}
\usepackage{txfonts}
\let\vec\mathbf{}
\usepackage{stfloats}
\usepackage{float}
\usepackage{cite}
\usepackage{cases}
\usepackage{subfig}
%\usepackage{xtab}
\usepackage{longtable}
\usepackage{multirow}
%\usepackage{algorithm}
\usepackage{amssymb}
%\usepackage{algpseudocode}
\usepackage{enumitem}
\usepackage{mathtools}
%\usepackage{eenrc}
%\usepackage[framemethod=tikz]{mdframed}
\usepackage{listings}
%\usepackage{listings}
\usepackage[latin1]{inputenc}
%%\usepackage{color}{   
%%\usepackage{lscape}
\usepackage{textcomp}
\usepackage{titling}
\usepackage{hyperref}
%\usepackage{fulbigskip}   
\usepackage{tikz}
\usepackage{graphicx}
\lstset{
  frame=single,
  breaklines=true
}
\let\vec\mathbf{}
\usepackage{enumitem}
\usepackage{graphicx}
\usepackage{siunitx}
\let\vec\mathbf{}
\usepackage{enumitem}
\usepackage{graphicx}
\usepackage{enumitem}
\usepackage{tfrupee}
\usepackage{amsmath}
\usepackage{amssymb}
\usepackage{mwe} % for blindtext and example-image-a in example
\usepackage{wrapfig}
\graphicspath{{figs/}}
\providecommand{\mydet}[1]{\ensuremath{\begin{vmatrix}#1\end{vmatrix}}}
\providecommand{\myvec}[1]{\ensuremath{\begin{bmatrix}#1\end{bmatrix}}}
\providecommand{\cbrak}[1]{\ensuremath{\left\{#1\right\}}}
\providecommand{\brak}[1]{\ensuremath{\left(#1\right)}}
\title{MATHEMATICS}
\author{SECTION A}
\date{\today}
\begin{document}

\maketitle

\begin{enumerate}
\section{Matrices}
\item $ A $ is a square matrix with $\mydet{A} = 4 $. Then find the value of  $\mydet{ A .\brak{\text{adj} A}}$.

\item For the matrix $A = \myvec{2 & 3  \\ 5 & 7} $ . Find $\brak{A + A'}$ and verify that it is a symmetric matrix.

\item Using elementary row transformations, find the inverse of the matrix
$\myvec{2 & -3 &5 \\ 3 & 2 & -4 \\ 1 & 1 & -2}$ 


\item Using matrices, solve the following system of linear equations :
\begin{align*}
 x+2y-3z=-4\\
 2x+3y+2z=2\\
 3x-3y-4z=11
\end{align*}


\item Using properties of determinants, find the value of $k $ if
$\mydet{x & y & x+y \\ y & x+y & x \\ x+y & x & y} =k\mydet{x^{3}+y^{3}}$.


\section{probability}
\item In a multiple choice examination with three possible answers for each of the five questions, what is the probability that a candidate would get four or more correct answers just by guessing ?

\item The probabilities of solving a specific problem independently by $A$ and $B$ are $\frac{1}{3}$ and $\frac{1}{5}$ respectively. If both try to solve the problem independently, find the probability that the problem is solved.

\item There are three coins. One is a coin having tails on both faces, another is a biased coin that comes up tails $70\%$ of the time and the third is an unbiased coin. One of the coins is chosen at random and tossed, it shows tail. Find the probability that it was a coin with tail on both the faces.

\section{Algebra}
\item Prove that :
\begin{align*}
\cos^{-1}\brak{\frac{12}{13}}+\sin^{-1}\brak{\frac{3}{5}}=\sin^{-1} \brak{\frac{56}{65}}
\end{align*}

\section{vector}
\item Find the vector equation of the plane which contains the line of intersection of the planes 
$ \overrightarrow{r}.\brak{\hat{i} + 2\hat{j}+3\hat{k}}-4=0$ , $\overrightarrow{r}.
\brak{2\hat{i} + \hat{j} - \hat{k}}+5=0$ and which is perpendicular to the plane $\overrightarrow{r}.\brak{5\hat{i} + 3\hat{j} - 6\hat{k}}+8=0$.

\item Find the vector equation of a line passing through the point
$\brak{2, 3, 2}$ and parallel to the line $\overrightarrow{r}=\brak{-2\hat{i} +3\hat{j}}\lambda\brak{2\hat{i} - 3\hat{j}+6\hat{k}}$.
Also, find the distance between these two lines.

\item Find the value of $\vec{x}$ such that the four points with position vectors,
$\vec{A}\brak{3\hat{i} + 2\hat{j} + \hat{k}}$, $\vec{B}\brak{4\hat{i} + x\hat{j} + 5\hat{k}}$, $\vec{C}\brak{4\hat{i} + 2\hat{j} -2 \hat{k}}$, $\vec{D}\brak{6\hat{i} + 5\hat{j} - \hat{k}}$ are coplanar.

\item Find the vector equation of the plane determined by the points $\vec{A}\brak{3, -1, 2}, \vec{B}\brak{5, 2, 4}$ and $\vec{C}\brak{-1, -1, 6}$. Hence, find the distance of the plane, thus obtained,from the origin.

\item Find the coordinates of the foot of the perpendicular $\vec{Q}$ drawn from $\vec{P}\brak{3, 2, 1}$ to the plane $2x - y + z + 1 = 0$. Also, find the distance $\vec{P}\vec{Q}$ and the image of the point $\vec{P}$ treating this plane as a mirror.

\section{Differentiation}


\item A ladder $13 m$ long is leaning against a vertical wall. The bottom of the ladder is dragged away from the wall along the ground at the rate of $2 cm/sec$. How fast is the height on the wall decreasing when the foot of the ladder is $5 m$ away from the wall ?


\item If $y=\brak{\log x}^{x}+x^{\log x},$ find $\frac{dy}{dx}$.


\item If $x = \sin t ,  y= \sin pt$ , prove that $\brak{1-x^{2}}\frac{d^{2}y}{d x^{2}} - x \frac{dy}{dx} + p^{2}y=0$.

\item Differentiate 
\begin{align*}
  \tan^{-1}\brak{\frac{\sqrt{1+x^{2}}-\sqrt{1-x^{2}}}{\sqrt{1+x^{2}}+\sqrt{1-x^{2}}}}
\end{align*} with respect to $\cos^{-1}x^{2}$.

\item Form the differential equation representing the family of curves $y = A \sin x $, by eliminating the arbitrary constant $A$ .

\section{Integration}

\item Find the particular solution of the differential equation $\frac{d y}{d x}=\frac{xy}{x^{2}+y^{2}}$, given that $y = 1$ when $x = 0$.

\item Solve the differential equation $\frac{dy} {dx} =1+x^{2}+y^{2}+x^{2}y^{2}$, given that $y = 1$ when $x = 0$.

\item Prove that $\int_{0}^{a} f \brak {x}d x = \int_{0}^{a} f\brak{a-x} d x$, and hence evaluate  $\int_{0}^{1} x^{2}\brak{1-x}^{n}dx$ .

\section{Linear forms}
\item Using integration, find the area of the region bounded by the parabola $y^{2}=4x$ and the circle $4x^{2}+4y^{2}=9$.

\item Using the method of integration, find the area of the region bounded by the lines $3x - 2y + 1 = 0, 2x + 3y - 21 = 0$ and $x - 5y + 9 = 0$.


\item Using integration, find the area of the region bounded by the line
$y = 3x + 2,$ the x-axis and the ordinates $x = - 2$ and $x = 1$.

\item Using integration, find the area of the smaller region bounded by the ellipse $\frac{x^{2}}{9}+\frac{y^{2}}{4}=1$ and the line $\frac{x}{3}+\frac{y}{2}=1$.

\section{Functions}
\item Let $A= R - \{2\}$ and $B = R - \{1\}$. If $f : A \rightarrow B$ is a function defined by $f\brak{x}=\frac{x-1}{x-2}$, show that $f$ is one-one and onto. Hence, find $f^{-1}$.

\item Show that the relation $S$ in the set $A = \{x \in Z : 0  \le x  \le 12\}$ given by $S= \{\brak{a, b} : a, b \in Z, a-b $ is divisible by $ 3 $ \} is an equivalence relation.

\section{Optimization}

\item A dietician wishes to mix two types of food in such a way that the vitamin contents of the mixture contains at least $8 \hspace{2pt}units$ of vitamin $A$ and $10\hspace{2pt}units$ of vitamin $C $. Food $I $ contains $2\hspace{2pt} units/kg$ of vitamin $A$ and $1\hspace{2pt} unit/kg$ of vitamin $C$. It costs \rupee$50 \hspace{2pt}per kg$ to produce food $I$. Food $II$ contains $1 \hspace{2pt}unit/kg$ of vitamin $A$ and $2\hspace{2pt} units/kg$ of vitamin $C$ and it costs \rupee $70\hspace{2pt} per kg$ to produce food $II$. Formulate this problem as a LPP to minimise the cost of a mixture that will produce the required diet. Also find the minimum cost.


\item A manufacturer produces nuts and bolts. It takes $1$ hour of work on machine $A$ and $3$ hours on machine $B$ to produce a package of nuts. It takes $3$ hours on machine $A$ and $1$ hour on machine $B$ to produce a package of bolts. He earns a profit of \rupee~35 per package of nuts and \rupee~14 per package of bolts. How many packages of each should be produced each day so as to maximise his profit, if he operates each machine for atmost $12$ hours a day ? Convert it into an LPP and solve graphically.





\end{enumerate}
\end{document}
